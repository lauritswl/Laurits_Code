% Options for packages loaded elsewhere
\PassOptionsToPackage{unicode}{hyperref}
\PassOptionsToPackage{hyphens}{url}
%
\documentclass[
]{article}
\usepackage{amsmath,amssymb}
\usepackage{iftex}
\ifPDFTeX
  \usepackage[T1]{fontenc}
  \usepackage[utf8]{inputenc}
  \usepackage{textcomp} % provide euro and other symbols
\else % if luatex or xetex
  \usepackage{unicode-math} % this also loads fontspec
  \defaultfontfeatures{Scale=MatchLowercase}
  \defaultfontfeatures[\rmfamily]{Ligatures=TeX,Scale=1}
\fi
\usepackage{lmodern}
\ifPDFTeX\else
  % xetex/luatex font selection
\fi
% Use upquote if available, for straight quotes in verbatim environments
\IfFileExists{upquote.sty}{\usepackage{upquote}}{}
\IfFileExists{microtype.sty}{% use microtype if available
  \usepackage[]{microtype}
  \UseMicrotypeSet[protrusion]{basicmath} % disable protrusion for tt fonts
}{}
\makeatletter
\@ifundefined{KOMAClassName}{% if non-KOMA class
  \IfFileExists{parskip.sty}{%
    \usepackage{parskip}
  }{% else
    \setlength{\parindent}{0pt}
    \setlength{\parskip}{6pt plus 2pt minus 1pt}}
}{% if KOMA class
  \KOMAoptions{parskip=half}}
\makeatother
\usepackage{xcolor}
\usepackage[margin=1in]{geometry}
\usepackage{color}
\usepackage{fancyvrb}
\newcommand{\VerbBar}{|}
\newcommand{\VERB}{\Verb[commandchars=\\\{\}]}
\DefineVerbatimEnvironment{Highlighting}{Verbatim}{commandchars=\\\{\}}
% Add ',fontsize=\small' for more characters per line
\usepackage{framed}
\definecolor{shadecolor}{RGB}{248,248,248}
\newenvironment{Shaded}{\begin{snugshade}}{\end{snugshade}}
\newcommand{\AlertTok}[1]{\textcolor[rgb]{0.94,0.16,0.16}{#1}}
\newcommand{\AnnotationTok}[1]{\textcolor[rgb]{0.56,0.35,0.01}{\textbf{\textit{#1}}}}
\newcommand{\AttributeTok}[1]{\textcolor[rgb]{0.13,0.29,0.53}{#1}}
\newcommand{\BaseNTok}[1]{\textcolor[rgb]{0.00,0.00,0.81}{#1}}
\newcommand{\BuiltInTok}[1]{#1}
\newcommand{\CharTok}[1]{\textcolor[rgb]{0.31,0.60,0.02}{#1}}
\newcommand{\CommentTok}[1]{\textcolor[rgb]{0.56,0.35,0.01}{\textit{#1}}}
\newcommand{\CommentVarTok}[1]{\textcolor[rgb]{0.56,0.35,0.01}{\textbf{\textit{#1}}}}
\newcommand{\ConstantTok}[1]{\textcolor[rgb]{0.56,0.35,0.01}{#1}}
\newcommand{\ControlFlowTok}[1]{\textcolor[rgb]{0.13,0.29,0.53}{\textbf{#1}}}
\newcommand{\DataTypeTok}[1]{\textcolor[rgb]{0.13,0.29,0.53}{#1}}
\newcommand{\DecValTok}[1]{\textcolor[rgb]{0.00,0.00,0.81}{#1}}
\newcommand{\DocumentationTok}[1]{\textcolor[rgb]{0.56,0.35,0.01}{\textbf{\textit{#1}}}}
\newcommand{\ErrorTok}[1]{\textcolor[rgb]{0.64,0.00,0.00}{\textbf{#1}}}
\newcommand{\ExtensionTok}[1]{#1}
\newcommand{\FloatTok}[1]{\textcolor[rgb]{0.00,0.00,0.81}{#1}}
\newcommand{\FunctionTok}[1]{\textcolor[rgb]{0.13,0.29,0.53}{\textbf{#1}}}
\newcommand{\ImportTok}[1]{#1}
\newcommand{\InformationTok}[1]{\textcolor[rgb]{0.56,0.35,0.01}{\textbf{\textit{#1}}}}
\newcommand{\KeywordTok}[1]{\textcolor[rgb]{0.13,0.29,0.53}{\textbf{#1}}}
\newcommand{\NormalTok}[1]{#1}
\newcommand{\OperatorTok}[1]{\textcolor[rgb]{0.81,0.36,0.00}{\textbf{#1}}}
\newcommand{\OtherTok}[1]{\textcolor[rgb]{0.56,0.35,0.01}{#1}}
\newcommand{\PreprocessorTok}[1]{\textcolor[rgb]{0.56,0.35,0.01}{\textit{#1}}}
\newcommand{\RegionMarkerTok}[1]{#1}
\newcommand{\SpecialCharTok}[1]{\textcolor[rgb]{0.81,0.36,0.00}{\textbf{#1}}}
\newcommand{\SpecialStringTok}[1]{\textcolor[rgb]{0.31,0.60,0.02}{#1}}
\newcommand{\StringTok}[1]{\textcolor[rgb]{0.31,0.60,0.02}{#1}}
\newcommand{\VariableTok}[1]{\textcolor[rgb]{0.00,0.00,0.00}{#1}}
\newcommand{\VerbatimStringTok}[1]{\textcolor[rgb]{0.31,0.60,0.02}{#1}}
\newcommand{\WarningTok}[1]{\textcolor[rgb]{0.56,0.35,0.01}{\textbf{\textit{#1}}}}
\usepackage{graphicx}
\makeatletter
\def\maxwidth{\ifdim\Gin@nat@width>\linewidth\linewidth\else\Gin@nat@width\fi}
\def\maxheight{\ifdim\Gin@nat@height>\textheight\textheight\else\Gin@nat@height\fi}
\makeatother
% Scale images if necessary, so that they will not overflow the page
% margins by default, and it is still possible to overwrite the defaults
% using explicit options in \includegraphics[width, height, ...]{}
\setkeys{Gin}{width=\maxwidth,height=\maxheight,keepaspectratio}
% Set default figure placement to htbp
\makeatletter
\def\fps@figure{htbp}
\makeatother
\setlength{\emergencystretch}{3em} % prevent overfull lines
\providecommand{\tightlist}{%
  \setlength{\itemsep}{0pt}\setlength{\parskip}{0pt}}
\setcounter{secnumdepth}{-\maxdimen} % remove section numbering
\ifLuaTeX
  \usepackage{selnolig}  % disable illegal ligatures
\fi
\IfFileExists{bookmark.sty}{\usepackage{bookmark}}{\usepackage{hyperref}}
\IfFileExists{xurl.sty}{\usepackage{xurl}}{} % add URL line breaks if available
\urlstyle{same}
\hypersetup{
  pdftitle={Tutorial\_1\_Simulation},
  pdfauthor={Laurits},
  hidelinks,
  pdfcreator={LaTeX via pandoc}}

\title{Tutorial\_1\_Simulation}
\author{Laurits}
\date{2024-02-01}

\begin{document}
\maketitle

\hypertarget{tutorial-1}{%
\section{Tutorial 1}\label{tutorial-1}}

\hypertarget{q1}{%
\subsection{Q1}\label{q1}}

\hypertarget{a-i}{%
\subsubsection{a) i}\label{a-i}}

We return Y when x greater than or equal to the CDF off y. 1. We check
if \$x \le \sum{p_0} \textless=\textgreater{} x
\le P(Y=0)\textless=\textgreater{} x \le 0.2 \$ if this is not true we
check for the second k. 2. We check if \$x \le \sum{p_1}
\textless=\textgreater{} x \le P(Y=0)+P(Y=1)\textless=\textgreater{} x
\le 0.2 + 0.3 \textless=\textgreater{} x \le 0.5 \$ if this is not true
we check for the third k. 3. We check if \$x \le \sum{p_2}
\textless=\textgreater{} x
\le P(Y=0)+P(Y=1)+P(Y=2)\textless=\textgreater{} x \le 0.2 + 0.3 + 0.4
\textless=\textgreater{} x \le 0.9 \$ if this is not true we check for
the fourth k. 4. We check if \$x \le \sum{p_3} \textless=\textgreater{}
x \le P(Y=0)+P(Y=1)+P(Y=2)+P(Y=3)\textless=\textgreater{} x \le 0.2 +
0.3 + 0.4 + 0.1\textless=\textgreater{} x \le 1 \$

If any of these statements were true, the algorithm ends, by returning
the Y. Statement 4 will always be true.

\hypertarget{a-ii}{%
\subsubsection{a) ii}\label{a-ii}}

The probability that Y = 2 is 0.4

\hypertarget{b-i}{%
\subsubsection{b) i}\label{b-i}}

The interval of X on which we will get Y = k is:
\[(\sum^{y}_{k=0}{p_{k-1}}:\sum^{y}_{k=0}{p_{k}}]\]

\hypertarget{b-ii}{%
\subsubsection{b) ii}\label{b-ii}}

So, the probability is:
\[\frac{\sum^{y}_{k=0}{p_{k}}-\sum^{y}_{k=0}{p_{k-1}}}{1-0}\] \#\# Q2
\#\#\# a) \textbf{Find the CDF:} F(i) = \(p(Y \le i)\) =
\(\sum^{i}_{k=1}{\frac{1}{k(k+1)}}\) =
\(\sum^{i}_{k=i}{\frac{1}{k}-\frac{1}{k+1}}\)

\hypertarget{b}{%
\subsubsection{b)}\label{b}}

\textbf{Find G() for \(x \ge 0\)} \(G(i + 1)\) = \(1-\frac{1}{i+1}\)

\hypertarget{c}{%
\subsubsection{c)}\label{c}}

\textbf{Find \(G^{-1}(x)\)} \(G(i+1) = 1-\frac{1}{i+1}\)
\(y=1-\frac{1}{x}\) \(1-y=\frac{1}{x}\) \(x=\frac{1}{1-y} = G^{-1}(y)\)

\hypertarget{d}{%
\subsubsection{d)}\label{d}}

Step 1. Generate X \textasciitilde Unif(0,1) Step 2. Let Y =
{[}1/(1-X){]} Step 3. Return Y

\hypertarget{q3}{%
\subsection{Q3}\label{q3}}

For the following integrals, please write out the pseudo code to
calculate them with the Monte Carlo integration:

\begin{Shaded}
\begin{Highlighting}[]
\FunctionTok{set.seed}\NormalTok{(}\DecValTok{2024}\NormalTok{)}
\end{Highlighting}
\end{Shaded}

\hypertarget{a}{%
\subsubsection{a)}\label{a}}

\begin{Shaded}
\begin{Highlighting}[]
\NormalTok{q3a }\OtherTok{\textless{}{-}} \ControlFlowTok{function}\NormalTok{(X)\{}
  \FunctionTok{exp}\NormalTok{(}\DecValTok{1}\NormalTok{)}\SpecialCharTok{\^{}}\NormalTok{(X}\SpecialCharTok{+}\NormalTok{X}\SpecialCharTok{\^{}}\DecValTok{2}\NormalTok{)}
\NormalTok{\}}
\end{Highlighting}
\end{Shaded}

\begin{Shaded}
\begin{Highlighting}[]
\CommentTok{\# integral[{-}2:2](e\^{}\{x+x\^{}2\})}
\NormalTok{from }\OtherTok{\textless{}{-}} \SpecialCharTok{{-}}\DecValTok{2}
\NormalTok{to }\OtherTok{\textless{}{-}} \DecValTok{2}
\NormalTok{total\_length }\OtherTok{\textless{}{-}}\NormalTok{ to}\SpecialCharTok{{-}}\NormalTok{from}
\NormalTok{N }\OtherTok{\textless{}{-}}  \DecValTok{100000}
\NormalTok{sim\_a }\OtherTok{\textless{}{-}} \FunctionTok{runif}\NormalTok{(N)}\SpecialCharTok{*}\NormalTok{total\_length}\SpecialCharTok{+}\NormalTok{from}
\NormalTok{sim\_y\_a }\OtherTok{\textless{}{-}} \FunctionTok{q3a}\NormalTok{(sim\_a)}

\NormalTok{integral\_approx }\OtherTok{\textless{}{-}}\NormalTok{ (total\_length}\SpecialCharTok{/}\NormalTok{N)}\SpecialCharTok{*}\FunctionTok{sum}\NormalTok{(}\FunctionTok{q3a}\NormalTok{(sim\_a))}
\FunctionTok{paste}\NormalTok{(}\StringTok{"The Monte Carlo integration is approximated too:"}\NormalTok{,}\FunctionTok{round}\NormalTok{(integral\_approx,}\DecValTok{2}\NormalTok{))}
\end{Highlighting}
\end{Shaded}

\begin{verbatim}
## [1] "The Monte Carlo integration is approximated too: 92.69"
\end{verbatim}

\hypertarget{b-1}{%
\subsubsection{b)}\label{b-1}}

\begin{Shaded}
\begin{Highlighting}[]
\NormalTok{h }\OtherTok{\textless{}{-}} \ControlFlowTok{function}\NormalTok{(x) \{}
    \FunctionTok{exp}\NormalTok{(}\DecValTok{1}\NormalTok{)}\SpecialCharTok{\^{}{-}}\NormalTok{(x}\SpecialCharTok{\^{}}\DecValTok{2}\NormalTok{)}
\NormalTok{\}}
\end{Highlighting}
\end{Shaded}

\begin{Shaded}
\begin{Highlighting}[]
\NormalTok{N}\OtherTok{=}\DecValTok{10000}
\NormalTok{Z }\OtherTok{\textless{}{-}} \FunctionTok{runif}\NormalTok{(N)}
\NormalTok{negative }\OtherTok{\textless{}{-}}  \DecValTok{1{-}1}\SpecialCharTok{/}\NormalTok{Z}
\NormalTok{positive }\OtherTok{\textless{}{-}}  \DecValTok{1}\SpecialCharTok{/}\NormalTok{Z}\DecValTok{{-}1}
\NormalTok{int\_positive }\OtherTok{\textless{}{-}} \FunctionTok{h}\NormalTok{(positive)}\SpecialCharTok{*}\DecValTok{1}\SpecialCharTok{/}\NormalTok{(Z}\SpecialCharTok{\^{}}\DecValTok{2}\NormalTok{)}
\NormalTok{int\_negative }\OtherTok{\textless{}{-}} \FunctionTok{h}\NormalTok{(negative)}\SpecialCharTok{*}\DecValTok{1}\SpecialCharTok{/}\NormalTok{(Z}\SpecialCharTok{\^{}}\DecValTok{2}\NormalTok{)}
\NormalTok{new\_z }\OtherTok{\textless{}{-}}\NormalTok{ int\_negative }\SpecialCharTok{+}\NormalTok{ int\_positive}
\FunctionTok{paste}\NormalTok{(}\StringTok{"The Monte Carlo integration is approximated too:"}\NormalTok{,}\FunctionTok{round}\NormalTok{(}\FunctionTok{mean}\NormalTok{(new\_z),}\DecValTok{2}\NormalTok{))}
\end{Highlighting}
\end{Shaded}

\begin{verbatim}
## [1] "The Monte Carlo integration is approximated too: 1.78"
\end{verbatim}

\textbf{How do we find out that we need to 1/Z\^{}2 to get the correct
scale?}

\hypertarget{c-1}{%
\subsubsection{c)}\label{c-1}}

\begin{Shaded}
\begin{Highlighting}[]
\NormalTok{h }\OtherTok{\textless{}{-}} \ControlFlowTok{function}\NormalTok{(x,y) \{}
    \FunctionTok{exp}\NormalTok{(}\DecValTok{1}\NormalTok{)}\SpecialCharTok{\^{}}\NormalTok{(x}\SpecialCharTok{+}\NormalTok{y)}\SpecialCharTok{\^{}}\DecValTok{2}
\NormalTok{\}}
\end{Highlighting}
\end{Shaded}

\begin{Shaded}
\begin{Highlighting}[]
\NormalTok{N}\OtherTok{=}\DecValTok{10000}
\NormalTok{X }\OtherTok{\textless{}{-}} \FunctionTok{runif}\NormalTok{(N)}
\NormalTok{Y }\OtherTok{\textless{}{-}} \FunctionTok{runif}\NormalTok{(N)}

\FunctionTok{paste}\NormalTok{(}\StringTok{"The Monte Carlo integration is approximated too:"}\NormalTok{,}\FunctionTok{round}\NormalTok{(}\FunctionTok{mean}\NormalTok{(}\FunctionTok{h}\NormalTok{(X,Y)),}\DecValTok{2}\NormalTok{))}
\end{Highlighting}
\end{Shaded}

\begin{verbatim}
## [1] "The Monte Carlo integration is approximated too: 4.92"
\end{verbatim}

\hypertarget{d-1}{%
\subsubsection{d)}\label{d-1}}

Is it possible to see the code of how prof expected this to be solved?

\begin{enumerate}
\def\labelenumi{\arabic{enumi})}
\setcounter{enumi}{3}
\tightlist
\item
\end{enumerate}

\begin{Shaded}
\begin{Highlighting}[]
\NormalTok{N }\OtherTok{\textless{}{-}} \DecValTok{10000}
\NormalTok{num\_iterations }\OtherTok{\textless{}{-}} \DecValTok{1000}
\NormalTok{results }\OtherTok{\textless{}{-}} \FunctionTok{vector}\NormalTok{(}\StringTok{"numeric"}\NormalTok{, }\AttributeTok{length =}\NormalTok{ num\_iterations)}

\ControlFlowTok{for}\NormalTok{ (i }\ControlFlowTok{in} \DecValTok{1}\SpecialCharTok{:}\NormalTok{num\_iterations) \{}
\NormalTok{  U }\OtherTok{\textless{}{-}} \FunctionTok{runif}\NormalTok{(N)}
\NormalTok{  result }\OtherTok{\textless{}{-}} \FunctionTok{cov}\NormalTok{(U, }\FunctionTok{exp}\NormalTok{(}\DecValTok{1}\NormalTok{)}\SpecialCharTok{\^{}}\NormalTok{U)}
\NormalTok{  results[i] }\OtherTok{\textless{}{-}}\NormalTok{ result}
\NormalTok{\}}
\FunctionTok{paste}\NormalTok{(}\StringTok{"Mean of Means:"}\NormalTok{,}\FunctionTok{mean}\NormalTok{(results))}
\end{Highlighting}
\end{Shaded}

\begin{verbatim}
## [1] "Mean of Means: 0.140906404247999"
\end{verbatim}

\begin{Shaded}
\begin{Highlighting}[]
\FunctionTok{paste}\NormalTok{(}\StringTok{"SD of Means:"}\NormalTok{,}\FunctionTok{sd}\NormalTok{(results))}
\end{Highlighting}
\end{Shaded}

\begin{verbatim}
## [1] "SD of Means: 0.001288881455618"
\end{verbatim}

\end{document}
